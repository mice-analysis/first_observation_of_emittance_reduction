\section{Detectors}
\label{Sect:Detectors}

In this section, the resolution and efficiency of the detector systems is
discussed. The reconstruction is validated both internally within each
detector system, using the detector's internal redundancy, and by extrapolating
reconstructed events between detectors.

\subsection{Tracker}

\textcolor{red}{References}

Two trackers, made up of five stations each with three planes of scintillating
fibres are used to reconstruct the momentum of incoming particles. Particles 
make helical trajectories whose radius and wavelength vary according to the
transverse and longitudinal momentum respectively. The radius of the helix is
given approximately by
\begin{equation}
\label{eq:scifi_helix_radius}
r = \frac{p_t}{q B_z}
\end{equation}
and the wavenumber by
\begin{equation}
\label{eq:scifi_helix_wavenumber}
k = \frac{q B_z}{p_z}.
\end{equation}

Fitting is performed in several steps. Electronics signals arising from adjacent
fibres are collected into clusters. The position of clusters in adjacent planes 
are collected to form a space point. A first-pass fit of a perfect helix to the 
space points is  used for noise rejection and to seed a second-pass fit using a 
Kalman filter.

If the reconstruction is well-understood, the path of the reconstructed 
trajectory should match the position of the clusters. The $\chi^2$ distribution
of reconstructed tracks is shown in fig. \ref{fig:chi2}.

\textcolor{red}{Add chi2 fit; add comment indicating chi2 fit is consistent.}

\begin{figure}[!tbh]
    \centering
    \includegraphics*[width=0.45\textwidth]{03-Detectors/Figures/HallProbes/hp_65.eps}
    \includegraphics*[width=0.45\textwidth]{03-Detectors/Figures/HallProbes/hp_67.eps}
    \includegraphics*[width=0.45\textwidth]{03-Detectors/Figures/HallProbes/hp_77.eps}
    \includegraphics*[width=0.45\textwidth]{03-Detectors/Figures/HallProbes/hp_79.eps}
    \includegraphics*[width=0.45\textwidth]{03-Detectors/Figures/HallProbes/hp_66.eps}
    \includegraphics*[width=0.45\textwidth]{03-Detectors/Figures/HallProbes/hp_72.eps}
    \caption{Hall probe readings for the two datasets across the entire period where data was taken.}
\label{fig:hall_probes}
\end{figure}

In order to accurately reconstruct tracks the field must be well known. If the
modelled field used in reconstruction differs from the actual field, a helix
can be found but the longitudinal and transverse momenta will scale according to
eq. (\ref{eq:scifi_helix_radius}) and eq. (\ref{eq:scifi_helix_wavenumber}). In 
order to accurately reconstruct the field correctly it is essential to 
understand the measured field accurately.

\begin{figure}[!tbh]
    \centering
    \includegraphics*[width=0.45\textwidth]{03-Detectors/Figures/HallProbes/bfield_vs_z_ssu.eps}
    \includegraphics*[width=0.45\textwidth]{03-Detectors/Figures/HallProbes/bfield_vs_z_ssd.eps}
    \caption{Hall probe readings compared to the field model used for 
             reconstruction and track extrapolation. Blue dashed lines show the
             position of the tracker stations.}
\label{fig:field_map}
\end{figure}


The field in the tracker region was monitored during the data taking period by
several Hall probes. The Hall probe measurement in the tracker region is shown
in fig. \ref{fig:hall_probes} for several run periods, including the data taking
periods under study in this paper. Reproducibility is demonstrated at better than
$10^{-3}$ T level.

The measured fields are shown overlayed with the modelled field in the tracker 
region in fig. \ref{fig:field_map}. There is some disagreement between the model
and the hall probe readings, particularly in the fringe field of the magnets.

\textcolor{red}{Proof that the efficiency is understood (and good);} probably something like 
number of 5 point, 4 point tracks and demonstration that this is consistent with
expected noise/dead channels? Look at straight tracks data?

\subsection{TOF}

Description of the TOFs and reconstruction (few words but mostly citation).

Proof that the resolution is understood - e.g. plane 0 vs plane 1 for each TOF station.

Proof that the efficiency is understood - e.g. number of plane 0 hits vs number 
of plane 1 hits for each TOF station.

\section{Global reconstruction}


