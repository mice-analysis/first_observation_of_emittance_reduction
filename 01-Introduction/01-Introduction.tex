\section{Introduction}
\label{Sect:Intro}

\subsection{Ionisation Cooling}

Ionization cooling \cite{ionization_cooling} is the only known technique that 
can cool a muon beam on a timescale competitive with the muon lifetime 
\cite{neutrino_factory} \cite{neutrino_factory_2} \cite{muon_collider} 
\cite{muon_collider_2}. Muon cooling has never been demonstrated previously. In 
ionization cooling, a beam is passed through an absorber causing energy to be 
lost due to ionization of atomic electrons. This yields a reduction in 
normalized transverse emittance. Multiple Coulomb scattering from atoms causes 
an increase in angular divergence of the beam, and hence emittance growth. The 
change in normalized RMS emittance $\varepsilon_\perp$ in distance $dz$ is 
\cite{ionization_cooling}
\begin{equation}
\label{eq:depsdz}
\frac{d\varepsilon_{\perp}}{dz} \approx 
    - \frac{\varepsilon_\perp}{\beta^2 E_\mu} \left<\frac{dE}{dz}\right> 
    + \frac{\beta_\perp (13.6\,\mathrm{MeV/c})^2}{2 \beta^3 E_\mu m_\mu X_0}
\end{equation}
where $\beta_\perp$ is the transverse optical Twiss function, $\beta c$, 
$E_\mu$, $m_\mu$ are the particle velocity, energy and mass, and $X_0$ is the 
radiation length. There exists an equilibrium RMS emittance $\varepsilon_{eqm}$
\begin{equation}
\varepsilon_{eqm} \approx \frac{1}{2m_\mu} \frac{13.6^2}{X_0} \frac{\beta_\perp}{\beta \left<dE/dz\right>}
\label{eq:eqm_emittance}
\end{equation}
at which $d\varepsilon_{\perp}/dz = 0$. If a beam with emittance below 
equilibrium is incident on an absorber, its emittance increases on passage 
through the absorber. Otherwise the emittance decreases. The Muon Ionisation
Cooling Experiment (MICE) collaboration seek to demonstrate ionization cooling 
for the first time.

\subsection{The Muon Ionisation Cooling Experiment}

\begin{figure*}[!tbh]
    \includegraphics*[width=0.9\textwidth]{01-Introduction/Figures/Step-4-labels.pdf}
    \caption{The MICE apparatus. \label{fig:Step4}}
\end{figure*}

MICE Step IV  \cite{mice} \cite{mice_step_iv} consists of a transfer line to 
bring particles from the ISIS synchrotron at Rutherford Appleton Laboratory to 
the cooling experiment.  The cooling equipment consists of a section of a 
solenoid focussing ionization cooling cell. Detectors, placed upstream and 
downstream of the emittance reduction apparatus, measure the momentum, position 
and species of particles entering and leaving the cooling channel, enabling the 
measurement of change in normalized beam emittance of the ensemble.
A schematic of the apparatus is shown in fig.\,\ref{fig:Step4}.

\subsection{MICE Muon Beam line}
Pions are created by dipping a titanium target into the ISIS proton synchrotron. 
The MICE Muon Beam line is a  transfer line that has been constructed to 
transport the resultant 
particles to the cooling apparatus \cite{beamline} \cite{pion_contamination} 
\cite{characterization}.  The incoming particle momentum can be selected by 
varying the field in a pair of dipoles. Higher magnetic field selects higher 
particle momentum. A series of tungsten and brass irises are positioned in the 
transfer line, enabling the selection of different emittances for the ensemble.

Up to around 100 particles are observed per second. MICE accumulates data in 
runs, each run consisting of a single experimental configuration and lasting of 
order hours. Several runs are taken for each solenoid configuration.  MICE has 
taken data over thousands of runs, with many different configurations.

\begin{figure}[!tbh]
    \centering
    \includegraphics*[width=0.8\textwidth]{03-Detectors/Figures/field_2017-02-7/bfield_vs_z}
    \caption{Modelled magnetic field for the configuration on the axis and with 
    160 mm horizontal displacement from the axis. The readings from Hall probes, 
    situated 160 mm from the beam axis, are shown. Dashed lines indicate 
    position of the tracker stations and absorber. A cusp near to the absorber
    position corresponds to the position of a flip in the field polarity from
    positive to negative. \label{fig:field}}
\end{figure}

\subsection{Cooling Channel}
The cooling channel consists of three superconducting solenoid modules \cite{SS} 
\cite{FC}. Spectrometer solenoid modules at the upstream and downstream end of 
the channel (SSU and SSD respectively) each generate a region of uniform 
field in which diagnostic trackers are situated and a matching region that 
transports the beam from the solenoid to the focus coil module (FC). The focus coil 
module, positioned between the solenoids, provides additional focussing to 
increase the angular divergence of the beam at the absorber, improving the 
amount of emittance reduction that can be achieved.

The effect of two absorbers are studied in this note, placed in the centre of 
the FC. The liquid hydrogen absorber is a 21 litre vessel. When filled, 
the absorber presents 350 mm 
of liquid Hydrogen along the experimental axis. Liquid hydrogen was chosen as an 
absorber material as it provides less multiple Coulomb  scattering for a given 
energy loss. Containment of
the Hydrogen is provided by a pair of thin Aluminium windows. An additional
pair of windows provided secondary containment in case of failure of the primary
containment windows.

The lithium hydride absorber is a 65 mm thick cylinder. The cylinder had a thin
coating of parylene to prevent ingress of water or oxygen. Configurations with 
no absorber installed at all and with just the liquid hydrogen containment 
vessel are also studied in this note.

\subsection{Diagnostic Apparatus}
Upstream of the cooling apparatus, two time-of-flight detectors (ToFs) 
\cite{tof} \cite{tof2} enable the measurement of particle velocity. A 
complementary velocity measurement is made by a threshold Cerenkov counter 
\cite{ckov}. Scintillating fibre 
trackers, positioned either side of the absorber module (TKU and TKD respectively), enable the measurement 
of particle position and momentum upstream and downstream of the absorber. 
Further downstream an additional ToF detector, a KLOE Light pre-shower detector 
and Electron Muon Ranger enable rejection of electron impurities.

The trackers consist of 5 stations \cite{tracker_hardware} 
\cite{tracker_software}. Each station consists of 3 views, each view rotated by 
120$^\circ$ with respect to the previous view. Each view consists of 2 layers of 
scintillating fibres. Gangs of 7 scintillating fibres are read out together by 
cryogenically operated Visible Light Photon Counters, enabling the position of 
incident particles to be measured. The 
trackers are situated in uniform fields such that particles make a helical 
path. The magnitude of the field is measured using Hall probes situated in the 
region of the tracker. By measuring the radius and pitch of the helix, the 
momentum of the particle can be deduced. The trackers have sufficient redundancy 
to enable the track reconstruction to be internally validated in order to 
estimate the efficacy of the reconstruction.

Each ToF consists of two planes. Each ToF plane is made up of a number of 
scintillator slabs. Photomultiplier tubes at either end of the ToF slabs produce 
a signal when particles pass through the ToF. The time at which muons pass 
through the apparatus can be measured with a resolution of 60\,ps.

\subsection{Operation of the Equipment}
In this paper the evolution of phase space density is reported for a single 
configuration of the cooling magnets,  `2017-02 7'. The 
cooling channel magnet set currents and the beam optical parameters assuming no
beta-beating in the upstream spectrometer solenoid are listed in table 
\ref{tab:magnet_parameters}. A  model of the magnetic field in this 
configuration is shown in fig. \ref{fig:field}.

The MICE Muon Beam line settings were varied to mimic different beam conditions. 
Results from four transfer line configurations are reported, with the 
accumulated muon sample having nominal emittances of 3 mm, 4 mm, 6 mm  and 10 mm 
at momenta around 140 MeV/c in the upstream spectrometer solenoid. These 
configurations are `3-140+M3-Test4`, `4-140+M3-Test1`, `6-140+M3-Test2` and `10-140+M3-Test4`
henceforth denoted `3-140', `4-140', `6-140' and `10-140' respectively. The 
corresponding run numbers and time during which the datasets were taken are 
listed in table \ref{tab:run_list}.

\begin{table}
\centering
\caption{Magnet parameters and other information for 2017-02 7. The cooling 
channel was powered in flip mode, meaning that SSU and the upstream coil in FC 
had positive polarity while the downstream coil in FC and SSD had negative 
polarity.
\label{tab:magnet_parameters}}
\begin{tabular}{|l|c|}
\hline
Setting                        & 2017-02 7    \\
\hline
$\beta_\perp$ at absorber [mm] & 660          \\
Nominal momentum [MeV/c]       & 140          \\
\hline
SSU Center Coil [A]            & 205.7        \\
SSU Match Coil2 [A]            & 168.25       \\
SSU Match Coil1 [A]            & 191.0        \\
\hline
FC Coil [A]                    & 129.24       \\
\hline
SSD Match Coil2 [A]            & 195.72       \\
SSD Center Coil [A]            & 144.0        \\
\hline
\end{tabular}
\end{table}

\begin{table}
\centering
\caption{List of datasets for the 2017-02 7 cooling channel
configuration, together with the beginning and end time of the data sets. Runs in bold
have been included in the analysis; the others may be included in the future.
\label{tab:run_list}}
\begin{tabular}{|l|cccc|}
\hline
                  & Full absorber & Empty absorber & No absorber & Lithium Hydride \\
\hline
3-140+M3-Test4    & \topmattersplitcell{9947 9956 \\9961 {\bf9971}}
                  & \topmattersplitcell{10019 10025 10031 \\10038 10044 10049 \\10054 10055 10056 \\10057 10058 10059 \\10060 10061 10062 \\ {\bf10069} 10075}
                  & {\bf10444} 10448 10458 
                  & 10478 {\bf10483} 10488 \\
\hline
4-140+M3-Test1    & \topmattersplitcell{9948 9950 \\9958 {\bf9962}} 
                  & \topmattersplitcell{10022 10028 10034 \\ 10040 10047 \\ {\bf10064} 10071} 
                  & \topmattersplitcell{{\bf10445} 10449 \\10453 10461} & \topmattersplitcell{10480 {\bf10484} \\10489}  \\
\hline
6-140+M3-Test2    & \topmattersplitcell{9949 9959 \\9960 9963 \\ {\bf9966} 9972 \\9976} 
                  & \topmattersplitcell{10023 10029 10036 \\ 10042 {\bf10051} 10066 \\ 10067 10073} 
                  & \topmattersplitcell{ {\bf10446} 10450 \\10454 10459 \\10460 10463} & 10481 {\bf10485} 10487 \\
\hline
10-140+M3-Test4   & \topmattersplitcell{9953 9964 \\9967 9969 \\ {\bf9970} } 
                  & \topmattersplitcell{10024 10026 10030 \\ 10035 10037 10041 \\ 10043 10048 {\bf10052} \\10053 10065 10068 \\10072 10074 10077} 
                  & \topmattersplitcell{{\bf10447} 10451 \\10452 10455} 
                  & \topmattersplitcell{10482 {\bf10486}\\ 10490} \\
\hline
Start of first run& 12/10/2017 12:10 & 20/10/2017 18:18 & 06/12/2017 19:54 & 11/12/2017 18:02 \\
End of last run   & 15/10/2017 10:52 & 23/10/2017 09:28 & 08/12/2017 09:28 & 12/12/2017 11:59 \\
\hline
\end{tabular}
\end{table}

\subsection{Simulation}
The cooling channel was modelled using various codes. Simulated particles based 
on a representative model of the pion yield from the target were transported 
through to the downstream edge of D2 using G4Beamline \cite{g4beamline}. 
Downstream of this region, MAUS \cite{maus} was used to model particle transport 
and the response of the MICE detectors to the incoming beam.

Some issues exist in the simulation, some of which have been eliminated by 
artificially tuning the simulated experiment:
\begin{itemize}
\item The momentum of the incoming beam has approximately 4 $\%$ discrepancy between
data and simulation. This has been tuned out of the incoming beam distributions
by modifying the simulated dipole field so that the field is about 4 $\%$ higher 
than suggested by measurements with a hall probe.
\item The ToF reconstruction of the simulation shows a mean residual of order 100 ps.
\item The modelled tracker density has been tuned so that the simulated $\chi^2/n_{dof}$ 
distributions are
consistent with data. The density was increased from the expected 1 g/cm$^2$ to 
2 g/cm$^2$.
\item The simulated extrapolated radial distribution of particles at the 
diffuser does not match the measured extrapolated distribution of particles 
(discussed below).
\end{itemize}

\subsection{Demonstration of Ionization Cooling}
This note explores the details of MICE. Because it is the first analysis to use the
full MICE system, it is necessary to validate the subsystems. To this end,
\begin{itemize}
\item{Section 2:} describes the sample selection used in the rest of the analysis.
\item{Section 3:} describes steps taken to validate the detector equipment for
events within the sample outlined previously
\item{Section 4:} describes steps taken to validate the cooling channel
\item{Section 5:} explains the cooling measurement
\end{itemize}