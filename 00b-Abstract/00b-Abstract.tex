\begin{quotation}

\noindent
Muon ionization cooling is a technique by which muon beam emittance may be reduced in
order to improve the beam transmission and luminosity. Ionization cooling is a key
component of proposed muon facilities such as the Muon Collider and Neutrino Factory
but until now has never been demonstrated in practice. 

In this paper, the effect of focussing of muons onto an energy absorber, using
a very high acceptance solenoid assembly, is described. The muon phase space distribution is
measured upstream and downstream of the focus. The muons are studied both with and without
an energy absorber. The emittance of the muon ensemble is shown to decrease in the presence of
an energy absorber and the phase space density is shown to increase, indicating that the 
beam has been successfully cooled.
\end{quotation}


