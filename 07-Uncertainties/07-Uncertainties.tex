\section{Uncertainties}

Discuss how we deal with various uncertainties. I imagine two classes of
uncertainty: (i) those that affect the measurement; (ii) those that affect the
model.

\subsection{Measurement Uncertainties}

\bf{author: Liu? Blackmore?}

The actual resolutions/etc have been introduced in the detectors section - so 
the job here is to make statistical/MC arguments about how they tie in to the 
data. E.g. how does TKD inefficiency affect emittance calculation?

\begin{itemize}
\item Detector resolution
\item Detector noise and efficiency
\item Magnetic field in reconstruction region
\end{itemize}

\subsection{Model Uncertainties}

\bf{author: Liu}

We measure a given emittance change and try to tie it into a model. The model
can be wrong because e.g. the fields were wrong, the absorber was wrong, etc. We
have following sources of uncertainty:

\begin{itemize}
\item Field uncertainty (alignment, current, etc)
\item Absorber uncertainty (thickness, density, etc)
\item Other material budget
\item Not sure, maybe if we are projecting upstream tracks we should include the
uncertainty on the upstream measurement? This is a bit circular...
\end{itemize}


