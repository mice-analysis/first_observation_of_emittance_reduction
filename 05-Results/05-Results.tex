\section{Calculation of Amplitude}

Having validated the performance of the cooling channel equipment the main 
results of the analysis are now presented. In this note, cooling is 
characterised by the change in particle amplitude.

For a beam with elliptical phase space contours, the 4D amplitude of the 
$i^{th}$ particle is defined by
\begin{equation}
\label{eq:amplitude}
A_i = \epsilon_n (\vec{u}_i - <\vec{u}>)^T \mathbf{V^{-1}} (\vec{u}_i - <\vec{u}>)
\end{equation}
where $\epsilon_n$ is the 4D normalised emittance, $\vec{u}$ is the 4D phase
space vector $(x, p_x, y, p_y)$, $<\vec{u}>$ is the mean of the 4D phase space
vector and $\mathbf{V}$ is the matrix of covariances with elements 
$v_{ij} = <u_i u_j> - <u_i u_j>$. When $A_i$ is $1/\varepsilon_n$, eq. \ref{eq:amplitude}
also defines the locus of points in space $u_i$ that make up the RMS hyper-ellipsoid
corresponding to the covariance matrix $\mathbf{V}$.

It can be shown that, in the linear approximation, $A_i$ is conserved 
\cite{amplitude_conservation}. For a
so-called `matched' beam, $A_i$ is on-average conserved even in the non-linear
approximation, below some dynamic aperture. This is because muons follow closed 
contours in phase space. A `matched' beam is one where the curves in phase space
are uniformly filled, so that if a muon moves to a portion of the curve
with a higher amplitude, another muon will move to replace it at the lower
amplitude. On average, the amplitude does not change.

Following the failure of coil M2D \cite{m2d_failure}, it was not possible to 
find an optics that had both a matched beam in SSU and the good transmission and
tight focussing necessary to observe cooling. In order to make the required
measurements, the optics team sacrificed the matched beam requirement and
accepted worse beam transmission. As a
consequence, a modified amplitude is defined that explicitly rejects the effects
of tails. The basic algorithm is

\begin{minipage}{0.8\textwidth}
\begin{verbatim}

while events in sample {
    calculate amplitudes
    remove highest amplitude event
    update covariance matrix
}

\end{verbatim}
\end{minipage}

This basic algorithm is biased; the calculation of the covariance matrix uses
the same sample as the amplitude calculation, so the covariance matrix will be
biased by statistical deviations in the sample. In order to avoid this bias,
the sample is split into two - a `ref'(erence) sample, used to calculate the
covariance matrix, and a `test' sample, used to calculate the amplitudes. The
`ref' and `test' samples are swapped so that no statistical power is lost
from the analysis.

The algorithm is also rather CPU intensive; the amplitude
calculation has to be repeated for every event, every time an event is removed,
so CPU time goes as $O(N^2)$. Rather than repeat the amplitude calculation for
every event, the data can be binned by amplitude. Events are removed per-bin; 
the amplitudes are recalculated; and the process is repeated until no events are 
removed from a given bin. The CPU time is observed to go as $O(log(N))$. Further 
optimisation may be possible. 

The full algorithm is outlined below.

\begin{minipage}{0.8\textwidth}
\begin{verbatim}

Split data into equal size ref_sample and test_sample
for cut_bin in reversed(ref_sample_bins) {
    calculate covariance matrix using ref_sample
    calculate amplitudes in ref_sample and bin
    do until no events are removed {
        loop over ref_sample {
            rebin ref_sample events
            if new bin >= cut_bin {
                remove events from ref_sample 
            }
            update covariance matrix
        } 
    }
    loop over test_sample {
        rebin test_sample events using covariance matrix previously calculated
        if new_bin >= cut_bin {
            remove events from test_sample
            store the amplitude of removed events
         }
     }
}
swap the ref_bin and test_bin designation and repeat

\end{verbatim}
\end{minipage}

A test analysis is shown in fig. \ref{fig:test_amplitude}, together with a 2D
section of some of the 4D RMS hyper-ellipsoids used to calculate the amplitude. 
Because the ellipses are calculated for each subsample, there are two series 
(rose-tinted and green-tinted). The total sample size was 10000 events.

A double peaked distribution is used to illustrate the resolving power of the 
technique. As the centre of the beam is approached, the ellipse calculation 
successfully centres on the brighter peak. This yields a double-peaked amplitude 
distribution.

\begin{figure}[!tbh]
    \centering
    \includegraphics*[width=\textwidth]{05-Results/Figures/TestAmplitudeBinned.pdf}
    \caption{Test amplitude distribution: double peaked distribution with
    ellipses used for calculation of $\mathbf{V}$ superimposed (left); and
    calculated amplitude distribution (right).\label{fig:test_amplitude}}
\end{figure}

\subsection{Effect of Detector Performance}

The resolution, inefficiency and impurity of the detectors will lead to 
inaccuracies in the measured amplitude distribution. Tracker resolution leads to 
movement of events from 
one bin to another due to changes in the measured phase space variables from the
actual position or momentum of the particle. Impurity and 
inefficiency leads to an artificial enhancement or depletion of a particular region
of phase space, which is assumed to deplete each bin without causing any bin
migrations.

\subsubsection{Simulation of Detector Effects}
The detector simulation has been used to estimate the magnitude of these effects 
and make a correction. A hybrid simulation was devised in order to provide a sufficiently
large sample of events that statistical fluctuations in the sample did not
dominate over the relatively small effect of bin migrations due to the detectors.
The upstream data sample, taken at TKU station 5, was smeared using a KDE routine 
\cite{kde} and then the smeared distribution was sampled in order to provide a 
large number of events (approximately $10^6$ muons). All events were assumed to be 
muons in station 5, which is consistent with the simulation and data. This 
sample was tracked from TKU through to TKD.

Several subsamples were drawn from these muons. The upstream reconstructed 
sample had the following requirements:
\begin{itemize}
\item exactly one TKU track;
\item TKU $\chi^2$ per degree of freedom less than 4;
\item estimated radial excursion of the reconstructed track from the beam axis 
in TKU less than 150 mm; and
\item TKU momentum between 135 and 145 MeV/c at the reference plane.
\end{itemize}
The downstream reconstructed sample had the following requirements:
\begin{itemize}
\item in the upstream reconstructed sample;
\item exactly one TKD track;
\item TKD $\chi^2$ per degree of freedom less than 4;
\item estimated radial excursion of the reconstructed track from the beam axis 
in TKD less than 150 mm; and
\item TKD momentum between 90 and 170 MeV/c at the reference plane.
\end{itemize}

In order to estimate inefficiency and purity, `truth' samples were considered,
corresponding to the events that should have been reconstructed. The upstream 
truth sample had the following requirements:
\begin{itemize}
\item in the upstream reconstructed sample;
\item MC truth muon in all 5 TKU stations; and
\item estimated radial excursion of the truth track from the beam axis in TKU 
less than 150 mm; and
\end{itemize}
The downstream truth sample had the following requirements:
\begin{itemize}
\item in the upstream truth sample;
\item MC truth muon in all 5 TKD stations;
\item estimated radial excursion of the truth track from the beam axis in TKD 
less than 150 mm; and
\item TKD true momentum between 90 and 170 MeV/c at the reference plane.
\end{itemize}

Each muon was recorded with its truth phase space variables $(x, p_x, y, p_y)$ 
and, if it was reconstructed, its reconstructed phase space variables, both
recorded at the tracker reference planes. The number of events surviving each 
cut is listed in tables \ref{tab:systematics_mc_cuts_summary_0_0},
\ref{tab:systematics_mc_cuts_summary_1_0},
\ref{tab:systematics_mc_cuts_summary_2_0} and
 \ref{tab:systematics_mc_cuts_summary_3_0}.

\let\splitcell\undefined

\newcommand{\splitcell}[2][c]{%
\begin{tabular}[#1]{@{}c@{}}#2\end{tabular}}


\begin{landscape}
\begin{table}
\centering
\caption{The upstream reconstructed simulated sample is listed. \label{tab:systematics_mc_cuts_summary_0_0}}
\begin{tabular}[pos]{l|cccc}
                                                   & \splitcell{\\Simulated\\2017-2.7\\3-140\\lH2\\empty\\Systematics\\tku base\\} & \splitcell{\\Simulated\\2017-2.7\\4-140\\lH2\\empty\\Systematics\\tku base\\} & \splitcell{\\Simulated\\2017-2.7\\6-140\\lH2\\empty\\Systematics\\tku base\\} & \splitcell{\\Simulated\\2017-2.7\\10-140\\lH2\\empty\\Systematics\\tku base\\} \\
\hline                                            
All Events                                         &  1095790  &  1095787  &  1095774  &  1095724  \\
\hline                                            
One track in TKU                                   &  1044432  &  1033482  &  1013371  &  958894  \\
TKU $\chi^2/dof$                                   &  1024814  &  1013368  &  994331  &  941404  \\
TKU fiducial volume                                &  1024334  &  1013136  &  993833  &  931880  \\
\hline                                            
TKU momentum                                       &  717107  &  704300  &  699576  &  649264  \\
\hline                                            
Upstream Sample                                    &  717107  &  704300  &  699576  &  649264  \\
\hline                                            

\end{tabular}
\end{table}
\end{landscape}


\let\splitcell\undefined

\newcommand{\splitcell}[2][c]{%
\begin{tabular}[#1]{@{}c@{}}#2\end{tabular}}


\begin{landscape}
\begin{table}
\centering
\caption{The downstream reconstructed simulated sample is listed. \label{tab:systematics_mc_cuts_summary_1_0}}
\begin{tabular}[pos]{l|cccc}
                                                   & \splitcell{\\Simulated\\2017-2.7\\3-140\\lH2\\empty\\Systematics\\tku base\\} & \splitcell{\\Simulated\\2017-2.7\\4-140\\lH2\\empty\\Systematics\\tku base\\} & \splitcell{\\Simulated\\2017-2.7\\6-140\\lH2\\empty\\Systematics\\tku base\\} & \splitcell{\\Simulated\\2017-2.7\\10-140\\lH2\\empty\\Systematics\\tku base\\} \\
\hline                                            
Upstream Sample                                    &  717107  &  704300  &  699576  &  649264  \\
\hline                                            
One track in TKD                                   &  658809  &  660413  &  630981  &  491533  \\
TKD $\chi^2/dof$                                   &  636698  &  644689  &  611802  &  475933  \\
TKD fiducial volume                                &  628695  &  630782  &  579498  &  423850  \\
TKD momentum                                       &  615429  &  623630  &  571445  &  418220  \\
\hline                                            
Downstream Sample                                  &  615429  &  623630  &  571445  &  418220  \\
\hline                                            

\end{tabular}
\end{table}
\end{landscape}


\let\splitcell\undefined

\newcommand{\splitcell}[2][c]{%
\begin{tabular}[#1]{@{}c@{}}#2\end{tabular}}


\begin{landscape}
\begin{table}
\centering
\caption{The extrapolated reconstructed simulated sample is listed. \label{tab:systematics_mc_cuts_summary_2_0}}
\begin{tabular}[pos]{l|cccc}
                                                   & \splitcell{\\Simulated\\2017-2.7\\3-140\\lH2\\empty\\Systematics\\tku base\\} & \splitcell{\\Simulated\\2017-2.7\\4-140\\lH2\\empty\\Systematics\\tku base\\} & \splitcell{\\Simulated\\2017-2.7\\6-140\\lH2\\empty\\Systematics\\tku base\\} & \splitcell{\\Simulated\\2017-2.7\\10-140\\lH2\\empty\\Systematics\\tku base\\} \\
\hline                                            
Upstream Sample                                    &  717107  &  704300  &  699576  &  649264  \\
\hline                                            
Muon in all TKU stations                           &  717107  &  704300  &  699576  &  649264  \\
TKU true fiducial volume                           &  717106  &  704292  &  699565  &  648924  \\
\hline                                            
Upstream MC True Sample                            &  717106  &  704292  &  699565  &  648924  \\
\hline                                            

\end{tabular}
\end{table}
\end{landscape}


\let\splitcell\undefined

\newcommand{\splitcell}[2][c]{%
\begin{tabular}[#1]{@{}c@{}}#2\end{tabular}}


\begin{landscape}
\begin{table}
\centering
\caption{The upstream truth simulated sample is listed. \label{tab:systematics_mc_cuts_summary_3_0}}
\begin{tabular}[pos]{l|cccc}
                                                   & \splitcell{\\Simulated\\2017-2.7\\3-140\\lH2\\empty\\Systematics\\tku base\\} & \splitcell{\\Simulated\\2017-2.7\\4-140\\lH2\\empty\\Systematics\\tku base\\} & \splitcell{\\Simulated\\2017-2.7\\6-140\\lH2\\empty\\Systematics\\tku base\\} & \splitcell{\\Simulated\\2017-2.7\\10-140\\lH2\\empty\\Systematics\\tku base\\} \\
\hline                                            
Upstream MC True Sample                            &  717106  &  704292  &  699565  &  648924  \\
\hline                                            
Muon in all TKD stations                           &  669924  &  669547  &  646052  &  513525  \\
TKD true fiducial volume                           &  660024  &  653147  &  605786  &  446155  \\
TKD true momentum                                  &  659526  &  652419  &  604340  &  442888  \\
\hline                                            
Downstream MC True Sample                          &  659526  &  652419  &  604340  &  442888  \\
\hline                                            

\end{tabular}
\end{table}
\end{landscape}


\let\splitcell\undefined

\subsubsection{Calculation of Correction}
Several amplitude distributions were generated from these samples, with $N_i$
events in the $i^{th}$ bin:
\begin{itemize}
\item $N_i^{t|t}$: number of events in the $i^{th}$ bin of the amplitude 
$A^{t|t}$ calculated from the truth sample using truth phase space variables;
\item $N_i^{r|t}$: number of events in the $i^{th}$ bin of the amplitude 
$A^{r|t}$ calculated from  the reconstructed sample using truth phase space 
variables; and
\item $N_i^{r|r}$: number of events in the $i^{th}$ bin of the amplitude 
$A^{r|r}$  calculated from the reconstructed sample using reconstructed phase 
space variables.
\end{itemize}
Each of these distributions was calculated in both the upstream region and the 
downstream region. A vector $\vec{E}$ and matrix $\mathbf{S}$ are defined relating 
the number of events in each sample according to
\begin{align}
N_i^{r|t} &= \sum_j S_{ij} N_j^{r|r} \\
N_i^{t|t} &=  E_{i} N_i^{r|t}
\end{align}
so that
\begin{equation}
N_i^{t|t} = E_{i} \sum_j S_{ij} N_{j}^{r|r}.
\end{equation}
Then
\begin{equation}
\label{eq:efficiency}
E_i = \frac{N_i^{t|t}}{N_i^{r|t}}.
\end{equation}
$Q_{ij}$ is defined as the number of events in the $i^{th}$ bin of the 
distribution of $A^{r|r}$ and the $j^{th}$ bin of the distribution of $A^{r|t}$.
Then
\begin{equation}
\sum_j Q_{ij} = N_i^{r|t}
\end{equation}
so
\begin{equation}
S_{ij}  = \frac{Q_{ij}}{N_j^{r|r}}.
\end{equation}
$\mathbf{S}$ and $\vec{E}$ are useful quantities because for the same 
RMS ellipse, they are independent of the actual number of events in
each amplitude bin.

\subsubsection{Migration Matrix Correction}

The corrections have been calculated for 3-140, 4-140, 6-140 and 10-140 beams
with no absorber. Deviations in the corrections in TKD may exist when an 
absorber is installed due to the lower beam momentum but these are assumed to be 
small.

The full migration matrix $\mathbf{S}$ is shown in fig. 
\ref{fig:crossing_probability_tku} for the upstream tracker and fig. 
\ref{fig:crossing_probability_tkd} for the downstream tracker. It is noted that 
while bin migrations are relatively common, migrations through more than one bin
are rather rare.

The probability of an event being reconstructed in the same amplitude bin as it
was found in truth, corresponding to the diagonal elements of $\mathbf{S}$, is
shown in fig. \ref{fig:crossing_probability_diagonal_tku} and 
\ref{fig:crossing_probability_diagonal_tkd} for upstream and downstream detectors
respectively. The correction is significant and becomes larger for
higher amplitudes. There is a significant cusp at 15 mm amplitude, visible
in all upstream and downstream samples but especially pronounced in the 10-140 
sample.

\begin{figure}[!tbh]
    \topmattersysplot{05-Results/Figures/systematics_summary}{upstream}
                     {crossing_probability_upstream}
                     {Probability of TKU reconstruction in a certain amplitude 
                      bin compared to the true amplitude bin for 3-140 (top 
                      left); 4-140 (top right); 6-140 (bottom left) and 10-140 
                      (bottom right) configurations.
                      \label{fig:crossing_probability_tku}}
\end{figure}

\begin{figure}[!tbh]
    \topmattersysplot{05-Results/Figures/systematics_summary}{downstream}
                     {crossing_probability_downstream}
            {Probability of TKD reconstruction in a certain amplitude bin compared 
             to the true amplitude bin for 3-140 (top left); 4-140 (top right); 
             6-140 (bottom left) and 10-140 (bottom right) configurations.
             \label{fig:crossing_probability_tkd}}
\end{figure}

\begin{figure}[!tbh]
    \topmattersysplot{05-Results/Figures/systematics_summary}{upstream}
                     {crossing_probability_all_upstream_migration_matrix_graph}
            {Probability of remaining in the true amplitude bin after 
             reconstruction in TKU for 3-140 (top left); 4-140 (top right); 
             6-140 (bottom left) and 10-140 (bottom right) configurations.
             \label{fig:crossing_probability_diagonal_tku}}
\end{figure}

\begin{figure}[!tbh]
    \topmattersysplot{05-Results/Figures/systematics_summary}{downstream}
                     {crossing_probability_all_downstream_migration_matrix_graph}
            {Probability of remaining in the true amplitude bin after 
             reconstruction in TKD for 3-140 (top left); 4-140 (top right); 
             6-140 (bottom left) and 10-140 (bottom right) configurations.
             \label{fig:crossing_probability_diagonal_tkd}}
\end{figure}

\subsubsection{Inefficiency Correction}
\label{sec:inefficiency}

The inefficiency in TKU is less than $O(10^{-3})$, arising due to events that are 
reconstructed in the TKU fiducial volume but whose true trajectory falls outside 
the TKU fiducial volume. No correction is made in TKU.

There is a notable efficiency correction in TKD as shown in 
fig. \ref{fig:inefficiency_tkd}. The correction is large at low amplitudes, 
where inefficiency arises as scattering can be dominant over curvature of the 
track for tracks with low transverse momentum. There is also a large correction
at high amplitudes, where inefficiency can arise due to deformation of the helix
from field non-uniformity near to the end coils.

\begin{figure}[!tbh]
    \topmattersysplot{05-Results/Figures/systematics_summary}{downstream}
                     {inefficiency_all_downstream_pdf_ratio_graph}
            {TKD efficiency correction for 3-140 (top left); 4-140 (top right); 
             6-140 (bottom left) and 10-140 (bottom right) configuratons.
             \label{fig:inefficiency_tkd}}
\end{figure}

\subsection{Uncertainties}

Measurement uncertainty is dominated by statistical uncertainty and uncertainty 
in the correction. The estimation of this uncertainty is described below.

\subsubsection{Statistical uncertainty}
In this note, the change in emittance of a sample of muons is studied. 
Statistical uncertainty in the change in emittance arises due to finite sampling 
of the scattering and straggling distribution in the absorber. Explicitly, the
uncertainty arising due to sampling of the input beam distribution is not
considered in this note. There is no statistical uncertainty in the upstream 
distribution, because it is exactly that set of events that have been selected
for analysis.

The statistical uncertainty in the downstream distribution is taken by 
considering the number of events in the $i^{th}$ bin upstream and the $j^{th}$ bin 
downstream,  $N^{us \cup ds}_{ij}$. The number of events is taken to follow a 
binomial distribution  where the number of trials is taken as the number of 
events in the $i^{th}$ bin upstream, $N^{us}_i$ and the probability of success is 
estimated as the ratio $N^{us \cup ds}_{ij}/N^{us}_i$. The uncertainty in 
$N^{us \cup ds}_{ij}$ is given by the width of the 
$68 \%$ confidence interval and the uncertainty in the downstream bin 
$N^{us}_{j}$ is given by summing uncertainties in $N^{us \, \cup \, ds}_{ij}$ over all 
upstream bins. Summation is done by adding in quadrature.

The statistical uncertainty in the cumulative distribution is found by adding in
quadrature the error on each bin in the probability distribution that contributes
to a given cumulative bin. The statistical error in the ratio distributions is
found by adding in quadrature the upstream and downstream error, 
normalised to the number of events in each bin.

Only a subset of the full dataset is included in this analysis. More data can
reduce the statistical uncertainties.

\subsubsection{Correction Uncertainty}
The correction procedure outlined above assumes perfect knowledge of the 
detector system. In reality this is not perfectly known. In order to understand
this uncertainty, imperfections are
introduced into the simulation of the detector system and the resulting change
in the correction is assigned as the uncertainty. The following imperfections
are introduced one-by-one and their effects on the correction are studied:

\begin{itemize}
\item Displacement in the horizontal plane through 3 mm
\item Rotation in the horizontal plane through 3 mrad
\item Mispowering of the Centre coil by 3 $\%$
\item Mispowering of the End1 coil by 5 $\%$
\item Mispowering of the End2 coil by 5 $\%$
\item Increasing the tracker glue density by 50 $\%$
\end{itemize}

The effect of each of these uncertainties on the correction coefficients is 
shown in fig. \ref{fig:crossing_probability_systematic_tku},
\ref{fig:crossing_probability_systematic_tkd} and 
\ref{fig:inefficiency_systematic_tkd}.

\begin{figure}[!tbh]
    \topmattersysplot{05-Results/Figures/systematics_summary}{upstream}
                     {crossing_probability_all_upstream_migration_matrix_multigraph}
        {Effect on diagonal terms of the TKU $\mathbf{S}$ matrix for 3-140 (top 
         left); 4-140 (top right); 6-140 (bottom left) and 10-140 (bottom right) 
         configuratons. \label{fig:crossing_probability_systematic_tku}}
\end{figure}

\begin{figure}[!tbh]
    \topmattersysplot{05-Results/Figures/systematics_summary}{downstream}
                     {crossing_probability_all_downstream_migration_matrix_multigraph}
        {Effect on diagonal terms of the TKD $\mathbf{S}$ matrix for 3-140 (top 
         left); 4-140 (top right); 6-140 (bottom left) and 10-140 (bottom right) 
         configuratons. \label{fig:crossing_probability_systematic_tkd}}
\end{figure}

\begin{figure}[!tbh]
    \topmattersysplot{05-Results/Figures/systematics_summary}{downstream}
                     {inefficiency_all_downstream_pdf_ratio_multigraph}
        {Effect on the efficiency $\vec{E}$ of varying the simulation geometry
         as compared to the reconstruction geometyr in  TKD for 3-140 (top 
         left); 4-140 (top right); 6-140 (bottom left) and 10-140 (bottom right) 
         configuratons. \label{fig:inefficiency_systematic_tkd}}
\end{figure}

It is noted that even with this large data set, statistical fluctuations in
the uncertainties are significant.

\clearpage

\subsection{Results}

The amplitude density and cumulative distributions are shown in fig. 
\ref{fig:amplitude_pdf_reco} and \ref{fig:amplitude_cdf_reco}. The ratio of the 
downstream and upstream density distributions is shown in fig. 
\ref{fig:pdf_ratio}. The ratio of the downstream and upstream cumulative
distributions is shown in fig. \ref{fig:cdf_ratio}. Where the ratio is greater
than 1, the density has increased on passing between the trackers. If this 
occurs in the beam core, then it indicates events have migrated from the tail of
the beam towards the core, which is indicative of cooling.

For all 3-140 settings, the number of events in the beam core is lower 
downstream than upstream indicating significant dilution of the beam core. The 
3-140 setting exhibits significant mismatch and this dominates the amplitude 
evolution.

For the 4-140 settings the number of events in the beam core is approximately
the same upstream and downstream. This setting is expected to be near to the
equilibrium emittance of both absorbers.

For the 6-140 and 10-140 setting, the events traversing no absorber or with an
empty hydrogen absorber do not exhibit a significant change in core density.
The settings with a filled liquid hydrogen absorber or lithium hydride do 
exhibit a change in core density indicating migration of tracks towards the beam
core when an absorber is installed. The 10-140 setting exhibits a greater 
increase in core density, at the expense of more loss in the tails due to 
scraping. The onset of scraping can be seen most clearly at above 40 mm.

\begin{figure}[!tbh]
    \centering
    \topmatterallplots{05-Results}{compare_amplitude_data}{amplitude_pdf_reco}
    {Distribution of amplitudes. The upstream distribution is shown by orange
    circles while the downstream distribution is shown by green triangles. Where
    significant, statistical uncertainty is represented by bars and systematic 
    uncertainty by hashed boxes}
\end{figure}

\begin{figure}[!tbh]
    \centering
    \topmatterallplots{05-Results}{compare_amplitude_data}{amplitude_cdf_reco}
    {Cumulative amplitude distribution. The upstream distribution is shown by orange
    circles while the downstream distribution is shown by green triangles. Where
    significant, statistical uncertainty is represented by bars and systematic 
    uncertainty by hashed boxes.}
\end{figure}

\begin{figure}[!tbh]
    \centering
    \topmatterallplots{05-Results}{compare_amplitude_both}{pdf_ratio}
    {Ratio of amplitude density. Data is shown in black and simulation is shown in
     red. Statistical uncertainty is represented by bars and systematic 
     uncertainty by hashed boxes.}
\end{figure}

\begin{figure}[!tbh]
    \centering
    \topmatterallplots{05-Results}{compare_amplitude_both}{cdf_ratio}
    {Ratio of amplitude cumulative distribution. Data is shown in black and 
     simulation is shown in red. Statistical uncertainty is represented by bars 
     and systematic uncertainty by hashed boxes.}
\end{figure}
