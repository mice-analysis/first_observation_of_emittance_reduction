\section{Sample Selection}
\label{Sect:Cuts-1}

\textcolor{red}{Update for revised cuts e.g. extrapolation cuts and global recon. List sample sizes.}

\subsection{Particle events}
The MICE data acquisition system was set to trigger if TOF1 received 
simultaneous hits in a given slab. The trigger enables readout of the 
diagnostics during a short trigger window. All data acquired during this period 
is associated together and known as a particle event. 

Signals in the same detector are associated into space points, for signals that 
are consistent with a particle passing through a given spatial region, and 
tracks, for signals that are consistent with a particle with a given momentum 
passing through a number of space points. The full reconstruction chain is
described in \cite{maus_paper}.

\subsection{Sample Selection}
Particle events are selected for analysis according to a number of different
criteria. Two samples are considered: the upstream sample is selected based on 
criteria in the upstream detector system only; the downstream sample is 
selected from the upstream sample, based on additional criteria in the 
downstream detector system. Because the upstream sample is selected based on
measurements in the upstream detector system, the upstream sample is 
independent of any stochastic processes occurring in the absorber.

The sample selection criteria are detailed in Table \ref{tab:sample_selection}
together with the number of events surviving each cut. The criteria are
described in detail below.

\subsection{Upstream Sample}
Cuts applied to the upstream sample are described below.

\begin{itemize}
\item{One TKU Track:} Events have exactly one track reconstructed in TKU.
\item{One TOF0 Space Point:} Events have exactly one space point reconstructed in
TOF0.
\item{One TOF1 Space Point:} Events have exactly one space point reconstructed in
TOF1.
\item{TKU Momentum:} Events are required to have momentum reconstructed by TKU 
between 135 and 145 MeV/c. The momentum of events in each sample is shown in 
fig. \ref{fig:tku_mom}.
\textcolor{red}{Discuss low momentum bulge?}
\item{TOF01 Time:} Pions and electrons in the momentum selection described above
have a quite different time-of-flight between TOF0 and TOF1 to muons. Events are 
required to have time-of-flight between TOF0 and TOF1 consistent with a muon in
order to reject this background. The time-of-flight of events in each sample is 
shown in fig. \ref{fig:tof01}. The velocity of particles upstream of the diffuser
is faster for thicker diffuser settings, in order to yield a muon sample with
momentum peaked around 140 MeV/c in TKU. In order to correctly reject impurities,
the TOF01 cut is different for different beamline settings, as listed in table 
\ref{tab:tof01_cut}
\item{TKU $\chi^2$ per degree of freedom:} The reconstructed $\chi^2$ per degree of freedom in TKU is 
required to be less than 10. The $\chi^2$ per degree of freedom of events in 
each sample is shown in fig. \ref{fig:tku_chi2}.
\end{itemize}

\begin{table}
\caption{Upper and lower bound of TOF cuts for different beamline settings.
\label{tab:tof01_cut}}
\centering
\begin{tabular}[pos]{l|rr}
Beamline & Lower       & Upper \\
         & Bound [ns]  & Bound [ns] \\
\hline
3-140    & 27.0          & 32.0 \\
6-140    & 27.0          & 31.0 \\
10-140   & 27.0          & 30.0 \\
\end{tabular}
\end{table}

\begin{figure}[!tbh]
    \centering
    \includegraphics*[width=0.45\textwidth]{analysis_plots/plots_3-140-full/tku_p.eps}
    \includegraphics*[width=0.45\textwidth]{analysis_plots/plots_3-140-empty/tku_p.eps}
    \includegraphics*[width=0.45\textwidth]{analysis_plots/plots_6-140-full/tku_p.eps}
    \includegraphics*[width=0.45\textwidth]{analysis_plots/plots_6-140-empty/tku_p.eps}
    \includegraphics*[width=0.45\textwidth]{analysis_plots/plots_10-140-full/tku_p.eps}
    \includegraphics*[width=0.45\textwidth]{analysis_plots/plots_10-140-empty/tku_p.eps}
    \caption{Momentum measured by TKU for 3-140 beam (top), 6-140 beam (middle) and 10-140 beam (bottom).
    The left hand column shows data with full absorber; the right hand column shows data with empty absorber.
    The black line shows all events; red shows events in the upstream sample; green shows events in the downstream sample.
\label{fig:tku_mom}}
\end{figure}

\begin{figure}[!tbh]
    \centering
    \includegraphics*[width=0.45\textwidth]{analysis_plots/plots_3-140-full/tof01.eps}
    \includegraphics*[width=0.45\textwidth]{analysis_plots/plots_3-140-empty/tof01.eps}
    \includegraphics*[width=0.45\textwidth]{analysis_plots/plots_6-140-full/tof01.eps}
    \includegraphics*[width=0.45\textwidth]{analysis_plots/plots_6-140-empty/tof01.eps}
    \includegraphics*[width=0.45\textwidth]{analysis_plots/plots_10-140-full/tof01.eps}
    \includegraphics*[width=0.45\textwidth]{analysis_plots/plots_10-140-empty/tof01.eps}
    \caption{Time-of-flight measured between TOF0 and TOF1 for 3-140 beam (top), 6-140 beam (middle) and 10-140 beam (bottom).
    The left hand column shows data with full absorber; the right hand column shows data with empty absorber.
    The black line shows all events; red shows events in the upstream sample; green shows events in the downstream sample.
\label{fig:tof01}}
\end{figure}

\begin{figure}[!tbh]
    \centering
    \includegraphics*[width=0.45\textwidth]{analysis_plots/plots_3-140-full/chi2_tku.eps}
    \includegraphics*[width=0.45\textwidth]{analysis_plots/plots_3-140-empty/chi2_tku.eps}
    \includegraphics*[width=0.45\textwidth]{analysis_plots/plots_6-140-full/chi2_tku.eps}
    \includegraphics*[width=0.45\textwidth]{analysis_plots/plots_6-140-empty/chi2_tku.eps}
    \includegraphics*[width=0.45\textwidth]{analysis_plots/plots_10-140-full/chi2_tku.eps}
    \includegraphics*[width=0.45\textwidth]{analysis_plots/plots_10-140-empty/chi2_tku.eps}
    \caption{$\chi^2$ distribution in TKU for 3-140 beam (top), 6-140 beam (middle) and 10-140 beam (bottom).
    The left hand column shows data with full absorber; the right hand column shows data with empty absorber.
    The black line shows all events; red shows events in the upstream sample; green shows events in the downstream sample.
    \label{fig:tku_chi2}}
\end{figure}

Events which do not meet these criteria are not considered for analysis at all.

\subsection{Downstream Sample}
Cuts applied to the downstream sample are described below.

\begin{itemize}
\item{In Upstream Sample:} Events are required to be in the upstream sample to
be considered in the downstream sample.
\item{One TKD Track:} Events have exactly one track reconstructed in TKD.
\item{TKD Momentum:} Events are required to have reconstructed momentum between 
100 and 200  MeV/c. The momentum as reconstructed by TKD of events in each 
sample is shown in fig. \ref{fig:tkd_mom}
\item{TKD $\chi^2$ per degree of freedom:} The reconstructed $\chi^2$ per degree of freedom in TKD is 
required to be less than 10. The $\chi^2$ per degree of freedom of events in 
each sample in TKD is shown in fig. \ref{fig:tku_chi2}.
\end{itemize}

\begin{figure}[!tbh]
    \centering
    \includegraphics*[width=0.45\textwidth]{analysis_plots/plots_3-140-full/tkd_p.eps}
    \includegraphics*[width=0.45\textwidth]{analysis_plots/plots_3-140-empty/tkd_p.eps}
    \includegraphics*[width=0.45\textwidth]{analysis_plots/plots_6-140-full/tkd_p.eps}
    \includegraphics*[width=0.45\textwidth]{analysis_plots/plots_6-140-empty/tkd_p.eps}
    \includegraphics*[width=0.45\textwidth]{analysis_plots/plots_10-140-full/tkd_p.eps}
    \includegraphics*[width=0.45\textwidth]{analysis_plots/plots_10-140-empty/tkd_p.eps}
    \caption{TKD momentum for 3-140 beam (top), 6-140 beam (middle) and 10-140 beam (bottom).
    The left hand column shows data with full absorber; the right hand column shows data with empty absorber.
    The black line shows all events; red shows events in the upstream sample; green shows events in the downstream sample.
\label{fig:tkd_mom}}
\end{figure}

\begin{figure}[!tbh]
    \centering
    \includegraphics*[width=0.45\textwidth]{analysis_plots/plots_3-140-full/chi2_tkd.eps}
    \includegraphics*[width=0.45\textwidth]{analysis_plots/plots_3-140-empty/chi2_tkd.eps}
    \includegraphics*[width=0.45\textwidth]{analysis_plots/plots_6-140-full/chi2_tkd.eps}
    \includegraphics*[width=0.45\textwidth]{analysis_plots/plots_6-140-empty/chi2_tkd.eps}
    \includegraphics*[width=0.45\textwidth]{analysis_plots/plots_10-140-full/chi2_tkd.eps}
    \includegraphics*[width=0.45\textwidth]{analysis_plots/plots_10-140-empty/chi2_tkd.eps}
    \caption{$\chi^2$ distribution in TKD for 3-140 beam (top), 6-140 beam (middle) and 10-140 beam (bottom).
    The left hand column shows data with full absorber; the right hand column shows data with empty absorber.
    The black line shows all events; red shows events in the upstream sample; green shows events in the downstream sample.
\label{fig:tkd_chi2}}
\end{figure}

Events which do not meet these downstream sample criteria are not considered
for analysis. They may either have collided with the cooling channel aperture 
and been lost (scraping), or 
they may have been not observed by the detectors (inefficiency). Systematic 
correction and uncertainty due to detector inefficiency is discussed in Section 
\cite{sec:inefficiency}.
