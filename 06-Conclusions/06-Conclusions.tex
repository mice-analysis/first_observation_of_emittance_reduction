\section{Conclusions}

An analysis has been presented of the evolution of amplitude in the presence of
absorbers. As this is the first analysis to use the full MICE system including
upstream and downstream detectors and fields, the system performance has been
explored in detail.

The experiment has been described. The data selection has been described. The
resolution of the trackers was discussed and the reconstructed tracker tracks
were compared with each other and the ToF detectors following extrapolation 
through the fields. 

The cooling channel was studied. Exceptional stability was demonstrated in the 
cooling channel magnets. The optical performance was studied and the extrapolated
upstream optical functions were shown to agree well with the downstream optical
functions. The energy loss in the absorber was shown to be stable and reasonable.
The beam distributions of the beam upstream and downstream was shown.

The algorithm for amplitude calculation was outlined and shown to work with a 
toy distribution. A correction routine was discussed to correct for tracker 
resolution and inefficiency. The evaluation of uncertainties on the measurement
was described.

The amplitude distributions were shown upstream and downstream of the absorber.
For low input emittances or settings without an absorber, the beam core was shown
to exhibit no increase in density while for high input emittances and settings
with an absorber installed the beam core was shown to exhibit an increase in
density. This is consistent with beam cooling. The amount of cooling obsereved 
was broadly consistent with simulation.
