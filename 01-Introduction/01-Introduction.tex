\section{Introduction}
\label{Sect:Intro}

\subsection{Ionisation Cooling}

Ionization cooling \cite{ionization_cooling} is the only known technique that 
can cool a muon beam on a timescale competitive with the muon lifetime 
\cite{neutrino_factory} \cite{neutrino_factory_2} \cite{muon_collider} 
\cite{muon_collider_2}. Muon cooling has never been demonstrated previously. In 
ionization cooling, a beam is passed through an absorber causing energy to be 
lost due to ionization of atomic electrons. This yields a reduction in 
normalized transverse emittance. Multiple Coulomb scattering from atoms causes 
an increase in angular divergence of the beam, and hence emittance growth. The 
change in normalized RMS emittance $\varepsilon_\perp$ in distance $dz$ is 
\cite{ionization_cooling}
\begin{equation}
\frac{d\varepsilon_{\perp}}{dz} \approx 
    - \frac{\varepsilon_\perp}{\beta^2 E_\mu} \left<\frac{dE}{dz}\right> 
    + \frac{\beta_\perp (13.6\,\mathrm{MeV/c})^2}{2 \beta^3 E_\mu m_\mu X_0}
\end{equation}
where $\beta_\perp$ is the transverse optical Twiss function, $\beta c$, 
$E_\mu$, $m_\mu$ are the particle velocity, energy and mass, and $X_0$ is the 
radiation length. There exists an equilibrium RMS emittance $\varepsilon_{eqm}$
\begin{equation}
\varepsilon_{eqm} \approx \frac{1}{2m_\mu} \frac{13.6^2}{X_0} \frac{\beta_\perp}{\beta \left<dE/dz\right>}
\label{eq:eqm_emittance}
\end{equation}
at which $d\varepsilon_{\perp}/dz = 0$. If a beam with emittance below 
equilibrium is incident on an absorber, its emittance increases on passage 
through the absorber. Otherwise the emittance decreases.

\subsection{The Muon Ionisation Cooling Experiment}

\begin{figure*}[!tbh]
    \includegraphics*[width=0.9\textwidth]{01-Introduction/Figures/Step-4-labels.pdf}
    \caption{The MICE apparatus. \label{fig:Step4}}
\end{figure*}

MICE Step IV  \cite{mice} \cite{mice_step_iv} consists of a transfer line to 
bring particles from the ISIS synchrotron at Rutherford Appleton Laboratory to 
the cooling experiment.  The cooling equipment consists of a section of a 
solenoid focussing ionization cooling cell. Detectors, placed upstream and 
downstream of the emittance reduction apparatus, measure the momentum, position 
and species of particles entering and leaving the cooling channel, enabling the 
measurement of change in normalized beam emittance of the ensemble.
A schematic of the apparatus is shown in fig.\,\ref{fig:Step4}.

\subsection{Transfer Line}
Pions are created by dipping a titanium target into the ISIS proton synchrotron. 
A dedicated transfer line has been constructed to transport the resultant 
particles to the cooling apparatus \cite{beamline} \cite{pion_contamination} 
\cite{characterization}.  The incoming particle momentum can be selected by 
varying the field in a pair of dipoles. Higher magnetic field selects higher 
particle momentum. A series of tungsten and brass irises are positioned in the 
transfer line, enabling the selection of different emittances for the ensemble.

Up to around 100 particles are observed per second. MICE accumulates data in 
runs, each run consisting of a single experimental configuration and lasting of 
order hours. Several runs are taken for each solenoid configuration.  MICE has 
taken data over thousands of runs, with many different configurations.

\begin{figure}[!tbh]
    \centering
    \includegraphics*[width=0.8\textwidth]{01-Introduction/Figures/bfield_vs_z.eps}
    \caption{Modelled magnetic field for the configuration on the axis and with 
    160 mm horizontal displacement from the axis. Hall probes, situated 160 mm from 
    the beam axis, show a 2 $\%$ discrepancy with the model. Dashed lines show 
    position of the tracker stations and absorber. \label{fig:field}}
\end{figure}

\subsection{Cooling Channel}
The cooling channel consists of three superconducting solenoid modules \cite{SS} 
\cite{FC}. Two spectrometer solenoid modules each generate a region of uniform 
field in which diagnostic trackers are situated and a matching region that 
transports the beam from the solenoid to the focus coil module. The focus coil 
module, positioned between the solenoids, provides additional focussing to 
increase the angular divergence of the beam at the absorber, improving the 
amount of emittance reduction that can be achieved.

The absorber was a 21 litre vessel. When filled, the absorber presents 350 mm 
of liquid Hydrogen along the experimental axis. Liquid hydrogen was chosen as an 
absorber material as it provides less multiple Coulomb  scattering for a given 
energy loss, due to the smaller electric charge of the nucleus. Containment of
the Hydrogen was provided by a pair of thin Aluminium windows. An additional
pair of windows provided secondary containment in case of failure of the primary
containment windows.

\subsection{Diagnostic Apparatus}
Upstream of the cooling apparatus, two time-of-flight detectors (TOFs) 
\cite{tof} \cite{tof2} enable the measurement of particle velocity, which is 
validated by a threshold Cerenkov counter \cite{ckov}. Scintillating fibre 
trackers, positioned either side of the absorber module, enable the measurement 
of particle position and momentum upstream and downstream of the absorber. 
Further downstream an additional TOF detector, a KLOE Light pre-shower detector 
and Electron Muon Ranger enable rejection of electron impurities.

The trackers consist of 5 stations \cite{tracker_hardware} 
\cite{tracker_software}. Each station consists of 3 views, each view rotated by 
120$^\circ$ with respect to the previous view. Each view consists of 2 layers of 
scintillating fibres. Gangs of 7 scintillating fibres are read out together by 
cryogenically operated Visible Light Photon Counters, enabling the position of 
incident particles to be measured with a resolution of 0.3 to 0.4\,mm. The 
trackers are situated in uniform 3\,T fields such that particles make a helical 
path. The magnitude of the field is measured using Hall probes situated in the 
region of the tracker. By measuring the radius and pitch of the helix, the 
momentum of the particle can be deduced. The trackers have sufficient redundancy 
to enable the track reconstruction to be internally validated in order to 
estimate the efficacy of the reconstruction. The uncertainty on the momentum of 
each track is around 1-2\,MeV/c.

Each TOF consists of two planes. Each TOF plane is made up of a number of 
scintillator slabs. Photomultiplier tubes at either end of the TOF slabs produce 
a signal when particles pass through the TOF. The time at which muons pass 
through the apparatus can be measured with a resolution of 60\,ps.

\subsection{Operation of the Equipment}
In this paper the evolution of phase space density is reported for a single 
configuration of the cooling magnets,  `2017-02 7'. The 
cooling channel magnet set currents and the beam optical parameters assuming no
beta-beating in the upstream spectrometer solenoid are listed in table 
\ref{tab:magnet_parameters}. A  model of the magnetic field in this 
configuration is shown in fig. \ref{fig:field}.

The transfer line settings were varied 
to mimic different beam conditions. Results from four transfer line 
configurations are reported, with the accumulated muon sample having nominal 
emittances of 3 mm, 4 mm, 6 mm  and 10 mm at momenta around 140 MeV/c in the 
upstream spectrometer solenoid. These configurations are denoted 
`3-140', `4-140', `6-140' and `10-140' respectively.

\begin{table}
\centering
\caption{Magnet parameters and other information for 2017-02 7. The cooling channel was powered in 
flip mode, meaning that SSU and the upstream coil in FC had positive polarity 
while the downstream coil in FC and SSD had negative polarity.
\label{tab:magnet_parameters}}

\begin{tabular}{|l|c|}
\hline
Setting                       & 2017-02 7    \\
\hline
Full absorber runs            & 9946 to 9972 \\
Full Start time               & 12/10/2017 11:59  \\
Full End time                 & 14/10/2017 12:20  \\
Empty absorber runs           & 10014 to 10077 \\
Empty Start time              & 18/10/2017 14:34  \\
Empty End time                & 23/10/2017 09:28  \\
Nominal FC $\beta_\perp$ [mm] & \textcolor{red}{WHAT BETA FUNCTION?}  \\
Nominal momentum [MeV/c]      & 140          \\
\hline
SSU Center Coil [A]           & 205.7        \\
SSU Match Coil2 [A]           & 168.25       \\
SSU Match Coil1 [A]           & 191.0        \\
\hline
FC Coil [A]                   & 129.24       \\
\hline
SSD Match Coil2 [A]           & 195.72       \\
SSD Center Coil [A]           & 144.0        \\
\hline
\end{tabular}
\end{table}

\subsection{Simulation}
The cooling channel was modelled using various codes. Simulated particles based on a representative model of the pion yield from the target were transported through to the upstream edge of TOF1 using G4Beamline \cite{g4beamline}. Downstream of this region, MAUS \cite{maus} was used to model particle transport and the response of the MICE detectors to the incoming beam.


