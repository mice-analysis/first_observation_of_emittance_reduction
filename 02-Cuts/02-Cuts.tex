\section{Sample Selection}
\label{Sect:Cuts-1}

The selection of data from the reconstructed sample is detailed below.

\subsection{Particle events}
The MICE data acquisition system was set to trigger if ToF1 received 
simultaneous hits in a given slab. The trigger enables readout of the 
diagnostics during a short trigger window. All data acquired during this period 
is associated together and known as a particle event. 

Signals in the same detector are associated into space points, for signals that 
are consistent with a particle passing through a given spatial region, and 
tracks, for signals that are consistent with a particle with a given momentum 
passing through a number of space points. The full reconstruction chain is
described in \cite{maus_paper}.

\subsection{Sample Selection}
Particle events are selected for analysis according to a number of different
criteria. Two samples are considered: the upstream sample is selected based on 
criteria in the upstream detector system only; the downstream sample is 
selected from the upstream sample, based on additional criteria in the 
downstream detector system.

The number of events surviving each sample selection criterion are listed for data in 
Tables \ref{tab:data_cuts_summary_0_0}, \ref{tab:data_cuts_summary_0_1}, 
\ref{tab:data_cuts_summary_1_0} and \ref{tab:data_cuts_summary_1_1}. The number
of surviving events are listed for simulation in Tables \ref{tab:mc_cuts_summary_0_0}, 
\ref{tab:mc_cuts_summary_0_1}, \ref{tab:mc_cuts_summary_1_0} and 
\ref{tab:mc_cuts_summary_1_1}. The criteria are
described in detail below. In the tables criteria are listed 
sequentially, in the order that the selection was made. The number of events 
surviving the selection and all preceding selections in the table is listed.


\newcommand{\splitcell}[2][c]{%
\begin{tabular}[#1]{@{}c@{}}#2\end{tabular}}


\begin{landscape}
\begin{table}
\centering
\caption{The upstream reconstructed data sample is listed.  Samples are listed for 3-140 and 4-140 datasets.\label{tab:data_cuts_summary_0_0}}
\begin{tabular}[pos]{l|cccccccc}
                                                   & \splitcell{\\2017-2.7\\3-140\\None\\} & \splitcell{\\2017-2.7\\3-140\\lH2\\empty\\} & \splitcell{\\2017-2.7\\3-140\\lH2\\full\\} & \splitcell{\\2017-2.7\\3-140\\LiH\\} & \splitcell{\\2017-2.7\\4-140\\None\\} & \splitcell{\\2017-2.7\\4-140\\lH2\\empty\\} & \splitcell{\\2017-2.7\\4-140\\lH2\\full\\} & \splitcell{\\2017-2.7\\4-140\\LiH\\} \\
\hline                                            
All Events                                         &  258683  &  172444  &  183035  &  240396  &  268081  &  217469  &   75102  &  218161  \\
\hline                                            
One space point in ToF1                            &  249235  &  166313  &  177064  &  231587  &  259651  &  210480  &   73147  &  211032  \\
One space point in ToF0                            &  196955  &  133090  &  146186  &  183288  &  205023  &  164195  &   60699  &  163990  \\
One track in TKU                                   &   74535  &   50901  &   56195  &   69895  &  140549  &  113793  &   41107  &  113726  \\
TKU $\chi^2/dof$                                   &   67654  &   46157  &   50464  &   63252  &  126403  &  101857  &   36187  &  102054  \\
TKU fiducial volume                                &   67282  &   45926  &   50165  &   62934  &  125951  &  101529  &   36051  &  101703  \\
\hline                                            
$t_{ToF1} - t_{ToF0}$                              &   39747  &   26920  &   28670  &   37505  &   78880  &   62821  &   22459  &   63222  \\
TKU momentum                                       &   14475  &    9548  &    9987  &   13143  &   32808  &   26233  &    9291  &   26562  \\
\hline                                            
Successful extrapolation to ToF0                   &   14473  &    9547  &    9981  &   13142  &   32798  &   26217  &    9275  &   26530  \\
Diffuser aperture cut                              &   13940  &    9317  &    9752  &   12703  &   32012  &   25728  &    9088  &   25936  \\
\hline                                            
Upstream Sample                                    &   13940  &    9317  &    9752  &   12703  &   32012  &   25728  &    9088  &   25936  \\
\hline                                            

\end{tabular}
\end{table}
\end{landscape}


\begin{landscape}
\begin{table}
\centering
\caption{The upstream reconstructed data sample is listed.  Samples are listed for 6-140 and 10-140 datasets.\label{tab:data_cuts_summary_0_1}}
\begin{tabular}[pos]{l|cccccccc}
                                                   & \splitcell{\\2017-2.7\\6-140\\None\\} & \splitcell{\\2017-2.7\\6-140\\lH2\\empty\\} & \splitcell{\\2017-2.7\\6-140\\lH2\\full\\} & \splitcell{\\2017-2.7\\6-140\\LiH\\} & \splitcell{\\2017-2.7\\10-140\\None\\} & \splitcell{\\2017-2.7\\10-140\\lH2\\empty\\} & \splitcell{\\2017-2.7\\10-140\\lH2\\full\\} & \splitcell{\\2017-2.7\\10-140\\LiH\\} \\
\hline                                            
All Events                                         &  258972  &  177328  &  283405  &  307300  &  398000  &  209994  &  374910  &  479187  \\
\hline                                            
One space point in ToF1                            &  250774  &  171124  &  275269  &  296958  &  376486  &  196606  &  356829  &  448579  \\
One space point in ToF0                            &  198333  &  132548  &  226008  &  229741  &  287657  &  144999  &  281242  &  331756  \\
One track in TKU                                   &  134951  &   90985  &  151587  &  158194  &  151577  &   76646  &  146733  &  176111  \\
TKU $\chi^2/dof$                                   &  120532  &   80629  &  131917  &  140694  &  135293  &   68248  &  128669  &  157056  \\
TKU fiducial volume                                &  119939  &   80227  &  131247  &  140059  &  129260  &   65159  &  122803  &  150219  \\
\hline                                            
$t_{ToF1} - t_{ToF0}$                              &   77702  &   51148  &   85624  &   90183  &   85564  &   42666  &   84236  &   97156  \\
TKU momentum                                       &   30987  &   20428  &   33700  &   35411  &   25371  &   12545  &   24512  &   28468  \\
\hline                                            
Successful extrapolation to ToF0                   &   30959  &   20396  &   33553  &   35274  &   24903  &   12226  &   22905  &   26900  \\
Diffuser aperture cut                              &   30320  &   20027  &   32943  &   34613  &   20403  &   10027  &   19234  &   22516  \\
\hline                                            
Upstream Sample                                    &   30320  &   20027  &   32943  &   34613  &   20403  &   10027  &   19234  &   22516  \\
\hline                                            

\end{tabular}
\end{table}
\end{landscape}


\let\splitcell\undefined

\newcommand{\splitcell}[2][c]{%
\begin{tabular}[#1]{@{}c@{}}#2\end{tabular}}


\begin{landscape}
\begin{table}
\centering
\caption{The downstream reconstructed data sample is listed.  Samples are listed for 3-140 and 4-140 datasets.\label{tab:data_cuts_summary_1_0}}
\begin{tabular}[pos]{l|cccccccc}
                                                   & \splitcell{\\2017-2.7\\3-140\\None\\} & \splitcell{\\2017-2.7\\3-140\\lH2\\empty\\} & \splitcell{\\2017-2.7\\3-140\\lH2\\full\\} & \splitcell{\\2017-2.7\\3-140\\LiH\\} & \splitcell{\\2017-2.7\\4-140\\None\\} & \splitcell{\\2017-2.7\\4-140\\lH2\\empty\\} & \splitcell{\\2017-2.7\\4-140\\lH2\\full\\} & \splitcell{\\2017-2.7\\4-140\\LiH\\} \\
\hline                                            
Upstream Sample                                    &   13940  &    9317  &    9752  &   12703  &   32012  &   25728  &    9088  &   25936  \\
\hline                                            
One track in TKD                                   &   13722  &    9112  &    9421  &   12413  &   31040  &   24720  &    8717  &   24927  \\
TKD $\chi^2/dof$                                   &   13208  &    8791  &    9113  &   12008  &   30041  &   23982  &    8472  &   24172  \\
TKD fiducial volume                                &   13135  &    8703  &    8893  &   11731  &   29347  &   23380  &    8207  &   23482  \\
TKD momentum                                       &   12945  &    8598  &    8838  &   11641  &   29028  &   23143  &    8146  &   23345  \\
\hline                                            
Downstream Sample                                  &   12945  &    8598  &    8838  &   11641  &   29028  &   23143  &    8146  &   23345  \\
\hline                                            

\end{tabular}
\end{table}
\end{landscape}


\begin{landscape}
\begin{table}
\centering
\caption{The downstream reconstructed data sample is listed.  Samples are listed for 6-140 and 10-140 datasets.\label{tab:data_cuts_summary_1_1}}
\begin{tabular}[pos]{l|cccccccc}
                                                   & \splitcell{\\2017-2.7\\6-140\\None\\} & \splitcell{\\2017-2.7\\6-140\\lH2\\empty\\} & \splitcell{\\2017-2.7\\6-140\\lH2\\full\\} & \splitcell{\\2017-2.7\\6-140\\LiH\\} & \splitcell{\\2017-2.7\\10-140\\None\\} & \splitcell{\\2017-2.7\\10-140\\lH2\\empty\\} & \splitcell{\\2017-2.7\\10-140\\lH2\\full\\} & \splitcell{\\2017-2.7\\10-140\\LiH\\} \\
\hline                                            
Upstream Sample                                    &   30320  &   20027  &   32943  &   34613  &   20403  &   10027  &   19234  &   22516  \\
\hline                                            
One track in TKD                                   &   28358  &   18603  &   30718  &   32458  &   15696  &    7666  &   15380  &   17857  \\
TKD $\chi^2/dof$                                   &   27293  &   17975  &   29763  &   31420  &   15069  &    7376  &   14873  &   17268  \\
TKD fiducial volume                                &   25969  &   17024  &   28626  &   30157  &   13099  &    6484  &   13519  &   15524  \\
TKD momentum                                       &   25727  &   16864  &   28459  &   30013  &   12971  &    6430  &   13461  &   15462  \\
\hline                                            
Downstream Sample                                  &   25727  &   16864  &   28459  &   30013  &   12971  &    6430  &   13461  &   15462  \\
\hline                                            

\end{tabular}
\end{table}
\end{landscape}


\let\splitcell\undefined

\newcommand{\splitcell}[2][c]{%
\begin{tabular}[#1]{@{}c@{}}#2\end{tabular}}


\begin{landscape}
\begin{table}
\centering
\caption{The upstream reconstructed simulated sample is listed.  Samples are listed for 3-140 and 4-140 datasets.\label{tab:mc_cuts_summary_0_0}}
\begin{tabular}[pos]{l|cccccccc}
                                                   & \splitcell{\\Simulated\\2017-2.7\\3-140\\None\\} & \splitcell{\\Simulated\\2017-2.7\\3-140\\lH2\\empty\\} & \splitcell{\\Simulated\\2017-2.7\\3-140\\lH2\\full\\} & \splitcell{\\Simulated\\2017-2.7\\3-140\\LiH\\} & \splitcell{\\Simulated\\2017-2.7\\4-140\\None\\} & \splitcell{\\Simulated\\2017-2.7\\4-140\\lH2\\empty\\} & \splitcell{\\Simulated\\2017-2.7\\4-140\\lH2\\full\\} & \splitcell{\\Simulated\\2017-2.7\\4-140\\LiH\\} \\
\hline                                            
All Events                                         &  159538  &  159322  &  159498  &  159950  &  201380  &   19237  &  201563  &  201596  \\
\hline                                            
One space point in ToF1                            &  141035  &  141130  &  141065  &  141655  &  181394  &   17388  &  181956  &  181798  \\
One space point in ToF0                            &  131638  &  131559  &  131411  &  131966  &  169328  &   16258  &  169852  &  169573  \\
One track in TKU                                   &   57449  &   64155  &   57118  &   57566  &  120987  &   11617  &  122058  &  121826  \\
TKU $\chi^2/dof$                                   &   51367  &   57960  &   51027  &   51448  &  109085  &   10471  &  109335  &  109429  \\
TKU fiducial volume                                &   48540  &   56027  &   48180  &   48721  &  107977  &   10377  &  108222  &  108362  \\
\hline                                            
$t_{ToF1} - t_{ToF0}$                              &   27503  &   33335  &   27352  &   27652  &   59105  &    5666  &   59271  &   59397  \\
TKU momentum                                       &    9585  &   10539  &    9460  &    9577  &   20649  &    1862  &   20979  &   20916  \\
\hline                                            
Successful extrapolation to ToF0                   &    9582  &   10519  &    9453  &    9571  &   20645  &    1860  &   20922  &   20848  \\
Diffuser aperture cut                              &    9351  &    9980  &    9263  &    9363  &   19680  &    1783  &   19986  &   19956  \\
\hline                                            
Upstream Sample                                    &    9351  &    9980  &    9263  &    9363  &   19680  &    1783  &   19986  &   19956  \\
\hline                                            

\end{tabular}
\end{table}
\end{landscape}


\begin{landscape}
\begin{table}
\centering
\caption{The upstream reconstructed simulated sample is listed.  Samples are listed for 6-140 and 10-140 datasets.\label{tab:mc_cuts_summary_0_1}}
\begin{tabular}[pos]{l|cccccccc}
                                                   & \splitcell{\\Simulated\\2017-2.7\\6-140\\None\\} & \splitcell{\\Simulated\\2017-2.7\\6-140\\lH2\\empty\\} & \splitcell{\\Simulated\\2017-2.7\\6-140\\lH2\\full\\} & \splitcell{\\Simulated\\2017-2.7\\6-140\\LiH\\} & \splitcell{\\Simulated\\2017-2.7\\10-140\\None\\} & \splitcell{\\Simulated\\2017-2.7\\10-140\\lH2\\empty\\} & \splitcell{\\Simulated\\2017-2.7\\10-140\\lH2\\full\\} & \splitcell{\\Simulated\\2017-2.7\\10-140\\LiH\\} \\
\hline                                            
All Events                                         &  211422  &   20294  &  211174  &  211571  &  296971  &   28407  &  297111  &  298029  \\
\hline                                            
One space point in ToF1                            &  190904  &   18330  &  190842  &  191258  &  268252  &   25632  &  268550  &  269256  \\
One space point in ToF0                            &  178199  &   17029  &  178120  &  178573  &  251126  &   23915  &  251552  &  251994  \\
One track in TKU                                   &  118921  &   11852  &  118688  &  119820  &  142143  &   12920  &  142204  &  143264  \\
TKU $\chi^2/dof$                                   &  105790  &   10478  &  104687  &  105813  &  127313  &   11579  &  127069  &  128045  \\
TKU fiducial volume                                &  103775  &   10310  &  102637  &  103692  &  108707  &   10118  &  108693  &  109849  \\
\hline                                            
$t_{ToF1} - t_{ToF0}$                              &   57928  &    5938  &   57627  &   58008  &   64554  &    6131  &   64627  &   65161  \\
TKU momentum                                       &   21072  &    2111  &   21065  &   21276  &   16545  &    1403  &   16546  &   16777  \\
\hline                                            
Successful extrapolation to ToF0                   &   21035  &    2104  &   20970  &   21165  &   15786  &    1351  &   14721  &   15110  \\
Diffuser aperture cut                              &   20498  &    1955  &   20577  &   20738  &   12513  &    1068  &   12327  &   12473  \\
\hline                                            
Upstream Sample                                    &   20498  &    1955  &   20577  &   20738  &   12513  &    1068  &   12327  &   12473  \\
\hline                                            

\end{tabular}
\end{table}
\end{landscape}


\let\splitcell\undefined

\newcommand{\splitcell}[2][c]{%
\begin{tabular}[#1]{@{}c@{}}#2\end{tabular}}


\begin{landscape}
\begin{table}
\centering
\caption{The downstream reconstructed simulated sample is listed.  Samples are listed for 3-140 and 4-140 datasets.\label{tab:mc_cuts_summary_1_0}}
\begin{tabular}[pos]{l|cccccccc}
                                                   & \splitcell{\\Simulated\\2017-2.7\\3-140\\None\\} & \splitcell{\\Simulated\\2017-2.7\\3-140\\lH2\\empty\\} & \splitcell{\\Simulated\\2017-2.7\\3-140\\lH2\\full\\} & \splitcell{\\Simulated\\2017-2.7\\3-140\\LiH\\} & \splitcell{\\Simulated\\2017-2.7\\4-140\\None\\} & \splitcell{\\Simulated\\2017-2.7\\4-140\\lH2\\empty\\} & \splitcell{\\Simulated\\2017-2.7\\4-140\\lH2\\full\\} & \splitcell{\\Simulated\\2017-2.7\\4-140\\LiH\\} \\
\hline                                            
Upstream Sample                                    &    9351  &    9980  &    9263  &    9363  &   19680  &    1783  &   19986  &   19956  \\
\hline                                            
One track in TKD                                   &    9038  &    9700  &    8854  &    8996  &   18902  &    1721  &   19049  &   19135  \\
TKD $\chi^2/dof$                                   &    8762  &    9406  &    8615  &    8733  &   18454  &    1674  &   18621  &   18675  \\
TKD fiducial volume                                &    8714  &    9304  &    8484  &    8587  &   18109  &    1639  &   18280  &   18220  \\
TKD momentum                                       &    8543  &    9103  &    8382  &    8481  &   17915  &    1617  &   18121  &   18111  \\
\hline                                            
Downstream Sample                                  &    8543  &    9103  &    8382  &    8481  &   17915  &    1617  &   18121  &   18111  \\
\hline                                            

\end{tabular}
\end{table}
\end{landscape}


\begin{landscape}
\begin{table}
\centering
\caption{The downstream reconstructed simulated sample is listed.  Samples are listed for 6-140 and 10-140 datasets.\label{tab:mc_cuts_summary_1_1}}
\begin{tabular}[pos]{l|cccccccc}
                                                   & \splitcell{\\Simulated\\2017-2.7\\6-140\\None\\} & \splitcell{\\Simulated\\2017-2.7\\6-140\\lH2\\empty\\} & \splitcell{\\Simulated\\2017-2.7\\6-140\\lH2\\full\\} & \splitcell{\\Simulated\\2017-2.7\\6-140\\LiH\\} & \splitcell{\\Simulated\\2017-2.7\\10-140\\None\\} & \splitcell{\\Simulated\\2017-2.7\\10-140\\lH2\\empty\\} & \splitcell{\\Simulated\\2017-2.7\\10-140\\lH2\\full\\} & \splitcell{\\Simulated\\2017-2.7\\10-140\\LiH\\} \\
\hline                                            
Upstream Sample                                    &   20498  &    1955  &   20577  &   20738  &   12513  &    1068  &   12327  &   12473  \\
\hline                                            
One track in TKD                                   &   18651  &    1724  &   18772  &   18944  &    9306  &     802  &    9521  &    9638  \\
TKD $\chi^2/dof$                                   &   18102  &    1673  &   18363  &   18423  &    9031  &     777  &    9300  &    9397  \\
TKD fiducial volume                                &   17046  &    1565  &   17637  &   17680  &    7960  &     677  &    8518  &    8586  \\
TKD momentum                                       &   16809  &    1539  &   17490  &   17549  &    7856  &     667  &    8422  &    8517  \\
\hline                                            
Downstream Sample                                  &   16809  &    1539  &   17490  &   17549  &    7856  &     667  &    8422  &    8517  \\
\hline                                            

\end{tabular}
\end{table}
\end{landscape}


\let\splitcell\undefined


\subsection{Upstream Sample}
Cuts applied to the upstream sample are described below.

\begin{itemize}
\item{One ToF1 Space Point:} Events have exactly one space point reconstructed in
ToF1. The number of space points per event observed in ToF1 is shown in fig. \ref{fig:tof1_n_sp}.
\item{One ToF0 Space Point:} Events have exactly one space point reconstructed in
ToF0. The number of space points per event observed in ToF1 is shown in fig. \ref{fig:tof0_n_sp}.
\item{One TKU Track:} Events have exactly one track reconstructed in TKU. The 
number of tracks observed in TKU is shown in fig. \ref{fig:tku_n_tk}.
\item{TKU $\chi^2$ per degree of freedom:} The reconstructed $\chi^2$ per degree of freedom in TKU is 
required to be less than 4. The $\chi^2$ per degree of freedom of events in 
each sample is shown in fig. \ref{fig:tku_chi2}.
\item{TKU fiducial cut:} Events are required to have a maximum radial excursion 
from the axis in TKU less than 150 mm. The maximum radial excursion of the track 
is estimated assuming a helical trajectory between tracker stations. The maximum 
radial excursion of events in each sample is shown in fig. \ref{fig:tku_max_r}.
\item{ToF01 Time:} Pions and electrons in the momentum selection described above
have a quite different time-of-flight between ToF0 and ToF1 to muons. Events are 
required to have time-of-flight between ToF0 and ToF1 consistent with a muon in
order to reject this background. The time-of-flight of events in each sample is 
shown in fig. \ref{fig:tof01}. The velocity of particles upstream of the diffuser
is faster for thicker diffuser settings, in order to yield a muon sample with
momentum peaked around 140 MeV/c in TKU. In order to correctly reject impurities,
the ToF01 cut is different for different beamline settings, as listed in table 
\ref{tab:tof01_cut}
\item{TKU Momentum:} Events are required to have momentum reconstructed by TKU 
between 135 and 145 MeV/c. The momentum of events in each sample is shown in 
fig. \ref{fig:tku_mom}.
%\item{Extrapolated $t_{ToF0}$ - Reconstructed $t_{ToF0}$ ($\delta t$):} The 
%measured time at ToF0 is compared with the time-of-flight expected by 
%extrapolating tracks from 
%TKU to ToF0, assuming a muon hypothesis. Those trajectories with 
%$-1.0 > \delta t > 1.5 ns$  are rejected from the analysis. The distribution of
%$\delta t_{01}$ is shown in fig. \ref{fig:delta_tof01}
\item{Successful extrapolation to ToF0:} Those events that are not successfully
extrpolated from TKU to ToF0 are rejected.
\item{Diffuser aperture cut:} Those events that have radius greater than 100 mm
at either the upstream or downstream face of the diffuser are rejected. The 
distribution of extrapolated radii are shown in fig. \ref{fig:diffuser_us} and
fig. \ref{fig:diffuser_ds} for upstream and downstream faces of the diffuser 
respectively.
\end{itemize}

\begin{table}
\caption{Upper and lower bound of ToF cuts for different beamline settings.
\label{tab:tof01_cut}}
\centering
\begin{tabular}[pos]{l|rr}
Beamline & Lower       & Upper \\
         & Bound [ns]  & Bound [ns] \\
\hline
3-140    & 1.5          & 6.5 \\
4-140    & 1.5          & 6.0 \\
6-140    & 1.5          & 5.5 \\
10-140   & 1.5          & 4.5 \\
\end{tabular}
\end{table}

\begin{figure}[!tbh]
    \centering
    \topmatterallplots{02-Cuts}{compare_cuts}{tof_tof1_n_sp_9_0}
    {The number of ToF1 space points for events that are accepted by every cut 
     except the requirement that there is 1 ToF1 space point.\label{fig:tof1_n_sp}}
\end{figure}

\begin{figure}[!tbh]
    \centering
    \topmatterallplots{02-Cuts}{compare_cuts}{tof_tof0_n_sp_9_0}
    {The number of ToF0 space points for events that are accepted by every cut 
      except the requirement that there is 1 ToF0 space point.\label{fig:tof0_n_sp}}
\end{figure}

\begin{figure}[!tbh]
    \centering
    \topmatterallplots{02-Cuts}{compare_cuts}{tku_n_tracks_8_0}
    {The number of tracks in TKU for events that are accepted by every upstream 
     cut except the requirement that there is 1 TKU track.\label{fig:tku_n_tk}}
\end{figure}

\begin{figure}[!tbh]
    \centering
    \topmatterallplots{02-Cuts}{compare_cuts}{tku_chi2_9_0}
    {$\chi^2$ per degree of freedom distribution in TKU for events that are 
    accepted by every upstream cut except the requirement on TKU $\chi^2$ per 
    degree of freedom.\label{fig:tku_chi2}}
\end{figure}

\begin{figure}[!tbh]
    \centering
    \topmatterallplots{02-Cuts}{compare_cuts}{tku_max_r_9_0}
    {Maximum radius in TKU for events that are accepted by every cut except the 
     maximum radius requirement.\label{fig:tku_max_r}}
\end{figure}

%\begin{figure}[!tbh]
%    \centering
%    \topmatterallplots{02-Cuts}{compare_cuts}{tof_delta_tof01_9_0}
%    {Discrepancy between the extrapolated time at ToF0 and the measured time at 
%    ToF0 ($\delta t_{01}$) for events that are accepted by every upstream cut 
%    except the requirement on $\delta t_{01}$. \textcolor{red}{BUG: bumps?}\label{fig:delta_tof01}}
%\end{figure}

\begin{figure}[!tbh]
    \centering
    \topmatterallplots{02-Cuts}{compare_cuts}{tof_tof01_9_0}
    {Time-of-flight measured between ToF0 and ToF1 for events that are accepted 
     by every upstream cut except for the time-of-flight cuts.\label{fig:tof01}}
\end{figure}

\begin{figure}[!tbh]
    \centering
    \topmatterallplots{02-Cuts}{compare_cuts}{tku_p_9_0}
    {Momentum measured by TKU for events that are accepted by every 
     upstream cut except for the TKU momentum cut. \label{fig:tku_mom}}
\end{figure}

\begin{figure}[!tbh]
    \centering
    \topmatterallplots{02-Cuts}{compare_cuts}{global_through_virtual_diffuser_us_r_9_0}
    {Extrapolated radius of tracks at the upstream face of the diffuser for 
     events that are accepted by every upstream cut except for the diffuser 
     radius cut. \textcolor{red}{BUG: 10-140 MC issue.} \label{fig:diffuser_us}}
\end{figure}

\begin{figure}[!tbh]
    \centering
    \topmatterallplots{02-Cuts}{compare_cuts}{global_through_virtual_diffuser_ds_r_9_0}
   {Extrapolated radius of tracks at the upstream face of the diffuser for 
    events that are accepted by every upstream cut except for the diffuser 
    radius cut. \textcolor{red}{BUG: 10-140 MC issue.} \label{fig:diffuser_ds}}
\end{figure}

Events which do not meet these criteria are not considered for analysis at all.

\subsubsection{Diffuser Geometry}

The anomaly between MC and data in diffuser radius is noted in 
fig. \ref{fig:diffuser_us} and fig. \ref{fig:diffuser_ds}. The physical layout of the 
diffusers and support structure has been studied in a reasonable amount of detail and checked 
to be consistent with the geometry implemented in  MAUS. The discrepancy still 
exists. It is noted that the discrepancy is in the region outside the upstream
sample cut.

\clearpage

\subsection{Downstream Sample}
Cuts applied to the downstream sample are described below.

\begin{itemize}
\item{In Upstream Sample:} Events are required to be in the upstream sample to
be considered in the downstream sample.
\item{One TKD Track:} Events have exactly one track reconstructed in TKD. The 
number of tracks in TKD for each event is shown in fig. \ref{fig:tkd_n_tracks}.
\item{TKD $\chi^2$ per degree of freedom:} The reconstructed $\chi^2$ per degree of freedom in TKD is 
required to be less than 4. The $\chi^2$ per degree of freedom of events in 
each sample in TKD is shown in fig. \ref{fig:tkd_chi2}.
\item{TKD Fiducial Volume:} Events are required to have a maximum radial excursion 
from the axis in TKD less than 150 mm. The maximum radial excursion of the track 
is estimated assuming a helical trajectory between tracker stations. The maximum 
radial excursion of events in each sample is shown in fig. \ref{fig:tkd_max_r}.
\item{TKD Momentum:} Events are required to have reconstructed momentum between 
90 and 170  MeV/c. The momentum as reconstructed by TKD of events in each 
sample is shown in fig. \ref{fig:tkd_mom}
\end{itemize}


\begin{figure}[!tbh]
    \centering
    \topmatterallplots{02-Cuts}{compare_cuts}{tkd_n_tracks_13_0}
    {The number of tracks measured in TKD for events that are accepted by every cut 
     except the requirement that there is 1 TKD track. \label{fig:tkd_n_tracks}}
\end{figure}

\begin{figure}[!tbh]
    \centering
    \topmatterallplots{02-Cuts}{compare_cuts}{tkd_chi2_13_0}
    {$\chi^2$ distribution in TKD for events that are accepted by every cut 
     except the requirement on $\chi^2$. \label{fig:tkd_chi2}}
\end{figure}

\begin{figure}[!tbh]
    \centering
    \topmatterallplots{02-Cuts}{compare_cuts}{tkd_max_r_13_0}
    {Maximum radius of tracks in TKD for events that are accepted by every cut 
     except the requirement on maximum radius. \label{fig:tkd_max_r}}
\end{figure}

\begin{figure}[!tbh]
    \centering
    \topmatterallplots{02-Cuts}{compare_cuts}{tkd_p_13_0}{Momentum in TKD,
    for events that are accepted by every downstream cut 
     except the TKD momentum requirement track. \label{fig:tkd_mom}}
\end{figure}

Events which do not meet these downstream sample criteria are not considered
for analysis in the downstream region. They may either have collided with the 
cooling channel aperture and been lost (scraping), or 
they may have been not observed by the detectors (inefficiency). Systematic 
correction due to detector inefficiency and the associated uncertainty is 
discussed in Section \ref{sec:inefficiency}.
